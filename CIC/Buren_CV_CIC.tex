% LaTeX resume using res.cls
\documentclass{res}
%\usepackage{helvetica} % uses helvetica postscript font (download helvetica.sty)
%\usepackage{newcent}   % uses new century schoolbook postscript font  
\usepackage{hanging}
%\usepackage[colorlinks=true,urlcolor=black,citecolor=black,linkcolor=black]{hyperref}
\usepackage[utf8]{inputenc}
\usepackage{color}
\usepackage{hyperref}
%\usepackage{bookmark}
\definecolor{darkblue}{rgb}{0.0,0.0,0.3}
\hypersetup{colorlinks,breaklinks,
	linkcolor=darkblue,urlcolor=darkblue,
	anchorcolor=darkblue,citecolor=darkblue}

%\usepackage{bookmark}

\setlength{\topmargin}{-0.6in}  % Start text higher on the page 
\setlength{\textheight}{9.8in}  % increase textheight to fit more on a page
\setlength{\headsep}{0.2in}     % space between header and text
\setlength{\headheight}{12pt}   % make room for header
\usepackage{fancyhdr}  % use fancyhdr package to get 2-line header
\renewcommand{\headrulewidth}{0pt} % suppress line drawn by default by fancyhdr
\lhead{\hspace*{-\sectionwidth}Alejandro Buren CV} % force lhead all the way left
\rhead{P\'agina \thepage}  % put page number at right
\cfoot{}  % the footer is empty
%\frenchspacing
\pagestyle{fancy} % set pagestyle for the document

% avoid word hyphenation at end of lines
\tolerance=1
\emergencystretch=\maxdimen
\hyphenpenalty=10000
\hbadness=10000

\begin{document} 
\thispagestyle{empty} % this page does not have a header
\name{Dr. Alejandro Daniel Buren}
%\address{222 South Main Street\\
%\address{\href{mailto:aburen@mun.ca}{aburen@mun.ca}\\
%\url{https://www.researchgate.net/profile/Alejandro_Buren}}



	
\address{ 
	\centerline{\href{mailto:aburen@mun.ca}{aburen@mun.ca}}\\
	\centerline{\url{https://www.researchgate.net/profile/Alejandro_Buren}}\\
	\centerline{\url{https://publons.com/researcher/1266176/alejandro-buren}}
%}	
%	Senior Fisheries Biologist\\
%	Ecofish Research Ltd.\\
%	\\	
%	Research Scientist\\
%Fisheries and Oceans Canada
%}
%(709) 772 - 4049\\
%alejandro.buren@dfo-mpo.gc.ca\\
%https://www.researchgate.net/profile/Alejandro\_Buren
}


\begin{resume}
%\vspace{0.1in}
\moveleft.5\sectionwidth\centerline{ }  


\section{EDUCACI\'ON}
\vspace{0.1in} 
 
\textbf{Doctor of Philosophy}. 2015 \\
Memorial University \\
Cognitive and Behavioural Ecology \\
Supervisor: William Montevecchi, Co-Supervisor: Mariano Koen-Alonso.
%\textit{Thesis title}: “The roles of capelin, climate, harp seals and fisheries in the failure of 2J3KL (northern) cod to recover”

\textbf{Master of Sciences}. 2007 \\
Memorial University \\
Cognitive and Behavioural Ecology \\
Supervisor: William Montevecchi, Co-Supervisor: Mariano Koen-Alonso.
%\textit{Thesis title}: ‘Modeling the link between prey availability and diet: common murre and capelin interaction during the breeding season’.

\textbf{Licenciado en ciencias biol\'ogicas}. 2004\\
Universidad Nacional de la Patagonia San Juan Bosco.\\
Supervisor: Enrique Crespo, Co-Supervisor: Mariano Koen-Alonso.
%\textit{Thesis title}: ‘Diet of the beaked skate, \textit{Dipturus chilensis}, off Patagonia during 2000-2001’



\section{EXPERIENCIA PROFESIONAL}
\vspace{0.1in}
%	\subsection{Professional, Research, and Teaching}
\textbf{Bi\'ologo Pesquero Superior,} Ecofish Research Ltd. Sep 2018-presente. T\'{i}tulo en ingl\'{e}s: Senior Fisheries Biologist
	
\textbf{Investigador Adjunto,} Ministerios de Pesquer\'ias y Oc\'eanos, Canad\'a.  2016-Sep 2018. T\'{i}tulo en ingl\'es: Research scientist (SE-RES-02). Fisheries and Oceans Canada (actualmente con licencia sin goce de sueldo)

\textbf{Jefe de Secci\'on Interino,} Ministerios de Pesquer\'ias y Oc\'eanos, Canad\'a. 2013-2018\\ Ocup\'e el cargo de Jefe de Secci\'on de Mam\'iferos Marinos regularmente cuando el titular se encontraba de viaje.

\textbf{Bi\'ologo Pesquero,} Ministerios de Pesquer\'ias y Oc\'eanos, Canad\'a. 2013-2016. T\'{i}tulo en ingl\'es: Aquatic sciences biologist, Fisheries and Oceans Canada.

\textbf{Jefe de C\'atedra,} Memorial University, Departmento de Biolog\'ia. 2012 \\ M\'etodos Cuantitavos en Biolog\'ia.  

\textbf{Facilitador de worskshops del lenguaje R,} Ministerios de Pesquer\'ias y Oc\'eanos, Canad\'a. 2012.

\textbf{Analista de datos,}  Ministerios de Pesquer\'ias y Oc\'eanos, Canad\'a. 2012\\
Contrato para analizar los efectos de una reducci\'on de muestreo sobre la distribuci\'on de tallas de especies blanco.

\textbf{Bi\'ologo, censo de focas de arpa,} Ministerios de Pesquer\'ias y Oc\'eanos, Canad\'a. 

\textbf{Dise\~{n}ador y desarrollador de bases de datos relacionales,} Ministerios de Pesquer\'ias y Oc\'eanos, Canad\'a.. 2007, 2008, 2009, 2011.

\textbf{Dise\~{n}ador y desarrollador de bases de datos relacionales,} Memorial University. 2007

\textbf{Miembro del Laboratorio de Mam\'iferos Marinos,} Centro Nacional Patag\'{o}nico (CENPAT), Argentina. 2001-2004
 
 \pagebreak

\section{PUBLICACIONES EN REVISTAS CON REFERATO} 

%\begin{hangparas}{.4in}{1}
Diecisiete publicaciones en revistas con referato, cuatro publicaciones como primer autor. 

Publicaciones en las revistas: PLoS ONE; Marine Ecology Progress Series; Endangered Species Research; ICES Journal of Marine Sciences; Polar Biology; Canadian Journal of Zoology; Journal of Sea Research; Estuarine, Coastal and Shelf Science; Aquatic Biology; Science E-letters; Marine Biology; Journal of Northwest Atlantic Fishery Science; Latin American Journal of Aquatic Mammals.
 
\section{EXPERIENCIA EN PROJECTOS DE INVESTIGACI\'ON}


Investigador y colaborador en m\'as de veinte proyectos de investigaci\'on con alcances regionales, nacionales e internacionales. 


Investigador responsable de cinco proyectos de alcance regional, nacional e internacional.  

Organismos financiadores: Fisheries and Oceans Canada, Natural Sciences and Engineering Research Council (Canad\'{a}), Natural Environment Research Council (Reino Unido), Arctic Council, Memorial University, Center of Expertise in Marine Mammalogy, Environmental Studies Research Fund, Species at Risk, International Polar Year, Agencia Nacional De Promoci\'on Cient\'ifica y Tecnol\'ogica, Parques Nacionales Argentina, Universidad Nacional de la Patagonia San Juan Bosco.



\section{PRESENTACIONES EN CONFERENCIAS Y WORKSHOPS}


M\'as de cincuenta presentaciones orales en conferencias, workshops y como orador invitado. Cinco presentaciones de p\'osters en conferencias.

Algunas conferencias relevantes: ICES Annual Science Conference,  International Marine Conservation Congress, Biennial Conference on the Biology of Marine Mammals, American Fisheries Society Annual Meeting, Arctic Frontiers, International Congress for Conservation Biology, NAMMCO SC Working Group on Marine Mammals and Fisheries Interactions ICES/NAFO/NAMMCO Symposium on the Role of Marine Mammals in the Ecosystem in the 21$^{st}$ Century, Pacific Seabird Group Conference.



\section{PREMIOS Y BECAS}
Doce premios y becas recibidas, por excelencia en el desarollo de actividades acad\'emicas. 

Entes emisores: Organizaci\'on de los Estados Americanos, Memorial University, University of British Columbia, Society for Marine Mammalogy, Universidad Nacional de la Patagonia San Juan Bosco, Ministerio de Educaci\'on de la Provincia del Chubut.



\section{REVISOR DE REVISTAS CON REFERATO}

Ecology, Canadian Journal of Fisheries and Aquatic Sciences, Global Change Biology, Proceedings of the Royal Society B, Journal of Applied Ecology, Marine Ecology Progress Series, ICES Journal of Marine Sciences, Journal of Avian Biology, Waterbirds


\end{resume}

\end{document}
