% LaTeX resume using res.cls
\documentclass{res}
%\usepackage{helvetica} % uses helvetica postscript font (download helvetica.sty)
%\usepackage{newcent}   % uses new century schoolbook postscript font  
\usepackage{hanging}
%\usepackage[colorlinks=true,urlcolor=black,citecolor=black,linkcolor=black]{hyperref}

\usepackage{color}
\usepackage{hyperref}
%\usepackage{bookmark}
\definecolor{darkblue}{rgb}{0.0,0.0,0.3}
\hypersetup{colorlinks,breaklinks,
	linkcolor=darkblue,urlcolor=darkblue,
	anchorcolor=darkblue,citecolor=darkblue}

%\usepackage{bookmark}

\setlength{\topmargin}{-0.6in}  % Start text higher on the page 
\setlength{\textheight}{9.8in}  % increase textheight to fit more on a page
\setlength{\headsep}{0.2in}     % space between header and text
\setlength{\headheight}{12pt}   % make room for header
\usepackage{fancyhdr}  % use fancyhdr package to get 2-line header
\renewcommand{\headrulewidth}{0pt} % suppress line drawn by default by fancyhdr
\lhead{\hspace*{-\sectionwidth}Alejandro Buren CV} % force lhead all the way left
\rhead{Page \thepage}  % put page number at right
\cfoot{}  % the footer is empty
%\frenchspacing
\pagestyle{fancy} % set pagestyle for the document

% avoid word hyphenation at end of lines
\tolerance=1
\emergencystretch=\maxdimen
\hyphenpenalty=10000
\hbadness=10000

\begin{document} 
\thispagestyle{empty} % this page does not have a header
\name{Alejandro Daniel Buren, PhD}
%\address{222 South Main Street\\
\address{Senior Fisheries Biologist\\
	Ecofish Research Ltd.\\
	\\	
	Research Scientist\\
Fisheries and Oceans Canada
}
%(709) 772 - 4049\\
%alejandro.buren@dfo-mpo.gc.ca\\
%https://www.researchgate.net/profile/Alejandro\_Buren}


\begin{resume}
\vspace{0.3in}
\moveleft.5\sectionwidth\centerline{ }  

\section{PERSONAL INFORMATION}

\parbox{0.5\textwidth}{ % First block
	\begin{tabbing} % Enables tabbing
		\hspace{3cm} \= \hspace{4cm} \= \kill % Spacing within the block
		
		{\bf Citizenship} \> Canadian \& Argentinean \\% Nationality
		
	    {\bf Office Phone} \> +1 (604) 608 6180 ext. 400 \\ % Home phone
		{\bf Cell Phone} \> +1 (604) 404 6643 \\ % Home phone
	
		{\bf Email} \> \href{mailto:aburen@mun.ca}{aburen@mun.ca}\\
		 
		{\bf ResearchGate} \> \url{https://www.researchgate.net/profile/Alejandro_Buren}
    \end{tabbing}
}
\hfill 

\section{EDUCATION}
\vspace{0.1in} 
 
\textbf{Doctor of Philosophy}. 2015 \\
Memorial University \\
Cognitive and Behavioural Ecology \\
Supervisor: William Montevecchi, Co-Supervisor: Mariano Koen-Alonso.\\
\textit{Thesis title}: “The roles of capelin, climate, harp seals and fisheries in the failure of 2J3KL (northern) cod to recover”

\textbf{Master of Sciences}. 2007 \\
Memorial University \\
Cognitive and Behavioural Ecology \\
Supervisor: William Montevecchi, Co-Supervisor: Mariano Koen-Alonso.\\
\textit{Thesis title}: ‘Modeling the link between prey availability and diet: common murre and capelin interaction during the breeding season’.

\textbf{Licentiate in biological sciences}. 2004\\
Universidad Nacional de la Patagonia San Juan Bosco.
(University of Patagonia) \\
Supervisor: Enrique Crespo, Co-Supervisor: Mariano Koen-Alonso.\\
\textit{Thesis title}: ‘Diet of the beaked skate, \textit{Dipturus chilensis}, off Patagonia during 2000-2001’


\section{LANGUAGES}
Advanced:
\begin{itemize}
	\item[] English
	\item[] Spanish (native language)
\end{itemize}
Beginner:
\begin{itemize}
	\item[] French
\end{itemize}
 
 

\section{PROFESSIONAL EXPERIENCE}
\vspace{0.1in}
%	\subsection{Professional, Research, and Teaching}
\textbf{Senior Fisheries Biologist,} Ecofish Research Ltd. Sep 2018-present
	
\textbf{Research scientist,} Fisheries and Oceans Canada. 2016-Sep 2018 (on leave of absence)

\textbf{Acting Section Head,} Fisheries and Oceans Canada. 2013-2018\\
Acted as Section Head of the Marine Mammals Section (NL Region) on a regular basis when the incumbent was away

\pagebreak

\textbf{Aquatic sciences biologist,} Fisheries and Oceans Canada. 2013-2016

\textbf{Primary instructor,} Memorial University, Biology Department. 2012 \\ Quantitative Methods in Biology. Graduate/undergraduate course. 

\textbf{R workshops facilitator,} Fisheries and Oceans Canada. 2012\\
Facilitate a series of introductory workshops to statistical language R.

\textbf{Data analyst,} Fisheries and Oceans Canada. 2012\\
Contract to assess the effects of sampling reduction in the length distribution of selected species in the spring and fall multi-species surveys. 

\textbf{Biologist, harp seal survey,} Fisheries and Oceans Canada, Marine Mammal Section. 2012\\
Biologist during the harp seal survey off southern Labrador and northeast Newfoundland.

\textbf{Database designer and developer,} Fisheries and Oceans Canada, Marine Mammal Section. 2007, 2008, 2009, 2011\\
Contracts to design, develop and populate the marine mammals relational database for the Newfoundland and Qu\'{e}bec Regions. 

\textbf{Database designer and developer,} Memorial University. 2007\\
Contract to design and develop relational database for a seabird research long term database. 

\textbf{Skipper of scientific platform,} Centro Nacional Patag\'{o}nico (CENPAT), Argentina. 2003-2004\\
Skipper of scientific platform dedicated to the study of small cetacean behavior and sample collection in the envelope of the Consultancy "Base line study and monitoring plan for marine mammals of the future National Park Monte Le\'{o}n". Argentinean National Parks Administration.\\

\textbf{Marine mammals laboratory member,} Centro Nacional Patag\'{o}nico (CENPAT), Argentina. 2001-2004\\

\textit{Note}: The CENPAT is a Research Institute which depends on the Argentinean National Research Council.

\textbf{Teaching assistant}
\begin{itemize}
	\item[] Memorial University, Biology Department
	\begin{itemize}
		\item[] 2009:
		\begin{itemize}
			\item[] Principles of Biology
			\item[] Biology for Earth Sciences
		\end{itemize}
	\end{itemize} 

	\begin{itemize}
    	\item[] 2006:
	    \begin{itemize}
		    \item[] Principles of Biology
     		\item[] Biology of Vertebrates
	    \end{itemize}
    \end{itemize} 

	\begin{itemize}
    	\item[] 2005:
    	\begin{itemize}
	    	\item[] Principles of Ecology
		    \item[] Principles of Biology
	    	\item[] Biology of Vertebrates
    	\end{itemize}
    \pagebreak
    \end{itemize} 
    \item[] University of Patagonia, Faculty of Natural Sciences
	\begin{itemize}
    	\item[] 2004:
	    \begin{itemize}
	    	\item[] Population Ecology. Joint graduate/undergraduate course
		    \item[] Mathematics I
	    \end{itemize}
    \end{itemize}    
    
   	\begin{itemize}
    	\item[] 2003:
    	\begin{itemize}
    		\item[] Population Ecology. Joint graduate/undergraduate course
    		\item[] Mathematics I
    	\end{itemize}
    \end{itemize}    
    
   	\begin{itemize}
    	\item[] 2000:
	    \begin{itemize}
    		\item[] Genetics and Evolution
	    \end{itemize}
    \end{itemize}      
\end{itemize}
 
\section{RESEARCH EXPERIENCE}
\vspace{0.1in}

{\bf Minimum realistic models of the Barents Sea and the Newfoundland and Labrador Shelf}. 2018-2020
\begin{itemize} % Use \item[] to prevent a bullet from appearing
	\item[] CoArc: A transatlantic innovation arena for sustainable development in the Arctic\\
	Principal Investigators: Dr Ulf Lindstr{\o}m, Dr Mariano Koen-Alonso
\end{itemize}

{\bf Ecophysiological responses of phocids to nutritional stress}. 2018-2020
\begin{itemize} % Use \item[] to prevent a bullet from appearing
	\item[] Fisheries and Oceans Canada and University of Pretoria (Southafrica)\\
	Investigators: Nico L\"ubcker, Dr. Alejandro Buren, Dr Nico de Bruyn, Dr Garry Stenson
\end{itemize}

{\bf Marine Mammal Expert Network}. 2017-2020
\begin{itemize} % Use \item[] to prevent a bullet from appearing
	\item[] Circumpolar Biodiversity Monitoring Program\\
	Conservation of Arctic Flora and Fauna\\
	Arctic Council
\end{itemize}

{\bf Natural Environment Research Council's Arctic Office (UK)}. 2018-2019
\begin{itemize}
	\item[] UK-Canada Bursaries Programme\\
	A Canary in a Cage: harp seals as indicators of climate change impacts in the Arctic\\
	Principal investigator: Dr James Grecian
\end{itemize}
{\bf Arctic Science Fund}. 2017-2019
\begin{itemize} % Use \item[] to prevent a bullet from appearing
	\item[] Fisheries and Oceans Canada\\
	Impacts of climate change on ringed seal growth and condition\\
	Principal Investigators: Dr Garry Stenson, Dr Alejandro Buren
\end{itemize}
{\bf Strategic Program for Ecosystem-Based Research and Advice}. 2017-2020
\begin{itemize} % Use \item[] to prevent a bullet from appearing
	\item[] Fisheries and Oceans Canada\\
	A predictive model of the environmental regulation of capelin\\
	Principal Investigators: Dr. Alejandro Buren, Dr. Paul Regular
\end{itemize}

\textbf{Fisheries Science and Ecosystem Research Program}. 2017-2019
\begin{itemize} % Use \item[] to prevent a bullet from appearing
	\item[] Fisheries and Oceans Canada\\
Principal Investigators: Dr. Pierre Pepin, Dr. Alejandro Buren; Dr. Mariano Koen-Alonso, Fran Mowbray, Dr. Paul Regular, Dr. Garry Stenson
\end{itemize}

\textbf{Natural Environment Research Council (UK)}. 2016-2018
\begin{itemize} % Use \item[] to prevent a bullet from appearing
	\item[] Can we detect changes in Arctic ecosystems?\\
Principal Investigators: Dr. C. Mahaffey, Dr. A. Tagliabue, Dr. R. Jeffreys, Prof. G. Wolff, Dr. A. Heath, Prof. R. Ganeshram, Dr. S. Smout, Dr. A. Yool, Dr. J. Hopkins, Dr. B. VanDongen, Dr. R. Stern \\
\url{https://arcticarise.wordpress.com/}
\end{itemize}

\pagebreak

\textbf{Strategic Initiatives}. 2018-2019
\begin{itemize} % Use \item[] to prevent a bullet from appearing
	\item[] Fisheries and Oceans Canada\\
Follow up on variability in energy contents of key forage species in the NW Atlantic: evidence for changes in ocean productivity?
Principal Investigators: Dr. Garry Stenson, Dr. Alejandro Buren
\end{itemize}

\textbf{Center of Expertise in Marine Mammalogy}. 2017
\begin{itemize} % Use \item[] to prevent a bullet from appearing
	\item[] Fisheries and Oceans Canada\\
Implementation of automated software for detection and counting of harp and hooded seals \\
Principal Investigator: Dr. Garry Stenson
\end{itemize}

\textbf{Center of Expertise in Marine Mammalogy}. 2017
\begin{itemize} % Use \item[] to prevent a bullet from appearing
	\item[] Fisheries and Oceans Canada\\
New efforts to monitor marine mammal bycatch in Atlantic Canada\\
Principal Investigator: Dr. Jack Lawson
\end{itemize}

\textbf{Strategic Initiatives}. 2016-2017
\begin{itemize} % Use \item[] to prevent a bullet from appearing
	\item[]Fisheries and Oceans Canada\\
Variability in energy contents of key forage species in the NW Atlantic: evidence for changes in ocean productivity?\\
Principal Investigators: Dr. Garry Stenson, Dr. Alejandro Buren
\end{itemize}

\textbf{Strategic Program for Ecosystem-Based Research and Advice}. 2017-2018
\begin{itemize} % Use \item[] to prevent a bullet from appearing
	\item[]Fisheries and Oceans Canada\\
Influence of plankton and environment on biomass and condition of capelin\\
Principal Investigator: Dr. César Fuentes-Yaco
\end{itemize}

\textbf{Strategic Program for Ecosystem-Based Research and Advice}. 2016-2020
\begin{itemize} % Use \item[] to prevent a bullet from appearing
	\item[]Fisheries and Oceans Canada\\
Spatial and temporal overlap of cetacean habitat, cumulative noise, and protected areas in the Northwest Atlantic Ocean\\
Principal Investigator: Dr. Hillary Moors-Murphy
\end{itemize}

\textbf{Laurentian Channel Marine Protected Area Monitoring}. 2014-2015
\begin{itemize} % Use \item[] to prevent a bullet from appearing
	\item[]Fisheries and Oceans Canada\\
Principal Investigator: Dr. Jack Lawson
\end{itemize}

\textbf{Environmental Studies Research Fund}. 2013-2014
\begin{itemize} % Use \item[] to prevent a bullet from appearing
	\item[]Fisheries and Oceans Canada\\
Mid-Labrador marine megafauna and acoustic survey\\
Principal Investigator: Dr. Jack Lawson\\
\url{http://www.esrfunds.org/home}
\end{itemize}

\textbf{Ecosystem Research Initiative, ERI-NEREUS Program}. 2007-2012
\begin{itemize} % Use \item[] to prevent a bullet from appearing
	\item[]Fisheries and Oceans Canada\\
Ecosystem Research Initiative (ERI) for Newfoundland Region\\
Principal Investigator: Dr. Mariano Koen-Alonso\\
\url{http://www.dfo-mpo.gc.ca/science/publications/article/2014/03-31-14-eng.html}
\end{itemize}

\textbf{Species at Risk Program}. 2006-2008
\begin{itemize} % Use \item[] to prevent a bullet from appearing
	\item[] Fisheries and Oceans Canada\\
Estimating predation on Atlantic Cod by Seals off Newfoundland and in the northern Gulf of St. Lawrence \\
Principal Investigator: Dr. Garry Stenson
\end{itemize}

\pagebreak

\textbf{International Polar Year}. 2007-2008
\begin{itemize} % Use \item[] to prevent a bullet from appearing
	\item[] Memorial University\\
How seabirds can help detect ecosystem change in the Arctic\\
Principal Investigator: Dr. William Montevecchi\\
\url{https://www.ec.gc.ca/api-ipy/default.asp?lang=En&n=7141EC8F-1}
\end{itemize}

\textbf{Natural Science And Engineering Research Council Strategic Project}. 2005-2007
\begin{itemize} % Use \item[] to prevent a bullet from appearing
	\item[] Memorial University\\
Importance of capelin (\textit{Mallotus villosus}) biology in sustaining trophic interactions in the Northwest Atlantic. \\
Principal Investigator: Dr. William Montevecchi.
\end{itemize}

\textbf{Agencia Nacional De Promoci\'{o}n Cient\'{i}fica y Tecnol\'{o}gica}. 2004-2006\\
(Argentinean National Agency for the Promotion of Science and Technology)
\begin{itemize} % Use \item[] to prevent a bullet from appearing
	\item[] University of Patagonia\\
Dynamics of the groups of dolphins affected by human activities and their inclusion in a marine protected areas management framework\\ 
Principal Investigator: Dr. Enrique Alberto Crespo
\end{itemize}

\textbf{Base line and monitoring study of marine mammals in National Park}. 2003-2004
\begin{itemize} % Use \item[] to prevent a bullet from appearing
	\item[] Marine Mammals Laboratory, CENPAT\\
Member of the Working Team of the Consultancy: Base line study and monitoring plan for marine mammals of the future National Park Monte Le\'{o}n\\
Principal Investigator: Dr. Enrique Alberto Crespo
\end{itemize}

\textbf{University of Patagonia}. 2002-2004
\begin{itemize} % Use \item[] to prevent a bullet from appearing
	\item[] Interactions between marine top predators and human activities in Patagonia\\
Principal Investigator: Dr. Enrique Alberto Crespo
\end{itemize}



\textbf{Agencia Nacional De Promoci\'{o}n Cient\'{i}fica y Tecnol\'{o}gica}. 2000-2002\\
(Argentinean National Agency for the Promotion of Science and Technology)
\begin{itemize} % Use \item[] to prevent a bullet from appearing
	\item[] University of Patagonia\\
Interactions between Fisheries and marine mammals in the Patagonian-Fueguinian coast\\
Principal Investigator: Dr. Enrique Alberto Crespo
\end{itemize}


\section{PUBLICATIONS} 
\subsection{Primary Publications}
%\vspace{0.1in}
\begin{hangparas}{.4in}{1}

J. Morgan, M. Koen-Alonso, R. Rideout, A.D. Buren \& D. Maddock Parsons. 2017. Growth and condition in relation to the lack of recovery of northern cod. ICES Journal of Marine Sciences 75: 631-641. DOI: 10.1093/icesjms/fsx166

C. Gomez,  J. Lawson, A-L Kouwenberg, H. Moors-Murphy, A.D. Buren, C. Fuentes-Yaco, E.  Marotte, Y. Wiersma \& T. Wimmer. 2017. Predicted distribution of whales at risk: identifying priority areas to enhance cetacean monitoring in the Northwest Atlantic Ocean. Endangered Species Research 32: 437-458. DOI: 10.3354/esr00823

E. Perry, G.B. Stenson \& A.D. Buren. 2017. Attendance and nursing patterns of harp seals in the harsh environment of the northwest Atlantic. Polar Biology 40: 151-160. DOI: 10.1007/s00300-016-1938-6

C. Gomez, J. Lawson, A. Wright, A.D. Buren, D. Tollit \& V. Lesage. 2016. A systematic review on the behavioural responses of wild marine mammals to noise: the disparity between science and policy. Canadian Journal of Zoology 94: 801–819. DOI: 10.1139/cjz-2016-0098

\pagebreak

G.B. Stenson, A.D. Buren \& M. Koen-Alonso. 2016. The impact of changing climate and abundance on reproduction in an ice-dependent species, the Northwest Atlantic harp seal, \textit{Pagophilus groenlandicus}. ICES Journal of Marine Science: Journal du Conseil 73: 250-262.

A.D. Buren, M. Koen-Alonso \& G.B Stenson. 2014. The role of harp seals, fisheries and food availability in driving the dynamics of northern cod. Marine Ecology Progress Series 511: 265-284.

A.D. Buren, M. Koen-Alonso, P. Pepin, F. Mowbray, B. Nakashima, G.B. Stenson, N. Ollerhead \& W.A. Montevecchi. 2014. Bottom-up regulation of capelin, a keystone forage species. PLoS ONE 9(2): e87589. doi:10.1371/journal.pone.0087589

A.D. Buren, M. Koen-Alonso \& W.A. Montevecchi. 2012. Predator diet and prey availability: common murre-capelin link in the Northwest Atlantic. Marine Ecology Progress Series 445: 25-35

KS Dwyer, A.D. Buren \& M. Koen-Alonso. 2010. Greenland halibut diet in the Northwest Atlantic from 1978-2003 as an indicator of ecosystem change. Journal of Sea Research 64: 436-445. 

L. McFarlane-Tranquilla, A. Hedd, C. Burke, W.A. Montevecchi, P.M. Regular, G.J. Robertson, L.A. Stapleton, S. Wilhelm, D.A. Fifield \& A.D. Buren. 2010. High Arctic sea ice conditions influence marine birds wintering in Low Arctic regions. Estuarine, Coastal and Shelf Science, 89: 97-106.

A. Hedd, D.A. Fifield, C. Burke, W.A. Montevecchi, L. McFarlane-Tranquilla, P.M. Regular, A.D. Buren \& G.J. Robertson. 2010. Seasonal shifts in the foraging niche of puffins \textit{Fratercula arctica} revealed by stable isotope ($\delta^{15}$N and $\delta^{13}$C) analyses. Aquatic Biology 9: 13-22.

W.A. Montevecchi, L. McFarlane-Tranquilla, A.D. Buren, C. Burke, D.A. Fifield, A. Hedd, P.M. Regular \& P.A. Smith. 2009. Effects of Geolocators in Studies of Avian Movement Ecology. Science E-letter \url{https://www.researchgate.net/publication/237628059_Effects_of_Geolocators_in_Studies_of_Avian_Movement_Ecology}

A. Hedd., P.M.Regular, W.A.Montevecchi, A.D.Buren, C.Burke \& D.A.Fifield. 2009. Going deep: common murres dive into frigid water for aggregated, persistent and slow-moving capelin. Marine Biology, 156: 741-751.

G. K. Davoren,, C. May, P. Penton, B. Reinfort, A.D. Buren, C. Burke, D. Andrews, W. A. Montevecchi, N. Record, B. deYoung, C. Rose-Taylor, T. Bell, J.T. Anderson, M. Koen-Alonso \& S. Garthe. 2008. An ecosystem-based research program for capelin (Mallotus villosus) in the Northwest Atlantic: overview and results. Journal of Northwest Atlantic Fishery Science, 39: 35-48.

M.F. Grandi, A.D. Buren, E.A. Crespo, N.A. García, G.M. Svendsen \& S.L. Dans. 2006. Record of a specimen of Shepherd’s beaked whale (\textit{Tasmacetus shepherdi}) from the coast of Santa Cruz, Argentina, with notes on age determination. Latin American Journal of Aquatic Mammals, 4(2): 97-100.
 

\end{hangparas}

\subsubsection{Accepted}
%\vspace{0.1in}
\begin{hangparas}{.4in}{1}
A.D. Buren, H.M. Murphy, A.T. Adamack, G.K. Davoren, M. Koen-Alonso, W.A. Montevecchi, F.K. Mowbray, P. Pepin, P.M. Regular, D. Robert, G.A. Rose, G.B. Stenson, D. Varkey. The collapse and continued low productivity of a keystone forage fish species. Marine Ecology Progress Series. To be published in 2019.
\end{hangparas}

\pagebreak

\subsubsection{In review}
%\vspace{0.1in}
\begin{hangparas}{.4in}{1}
	%\vspace{0.2in} 
K.E. Lewis, A.D. Buren, P.M. Regular, F.K. Mowbray, H.M. Murphy. Predicting capelin (\textit{Mallotus villosus}) biomass on the Newfoundland shelf. Submitted on July 2018 
to Marine Ecology Progress Series.

WA Montevecchi, K. Gerrow, A.D. Buren, G.K. Davoren, K.E. Lewis, M.W. Montevecchi, P.M. Regular. Pursuit-diving seabird buffers regime shift involving a three decadal decline of forage fish abundance and condition. Submitted on December 2018 to Marine Ecology Progress Series

\end{hangparas}
\subsubsection{In preparation}
%\vspace{0.1in}
\begin{hangparas}{.4in}{1}
%\vspace{0.2in} 

A.D. Buren, G.B. Stenson. Comparative diet analyses of four sympatric seal species, in waters off Newfoundland and Labrador Canada. To be submitted on July 2019 to Marine Mammal Science.


 
\end{hangparas}

\subsection{Secondary Publications}
\begin{hangparas}{.4in}{1}
	
G.B. Stenson, D. Wakeham, A.D. Buren \& M. Koen-Alonso. Density Dependent and Density Independent Factors Influencing Reproductive Rates in Northwest Atlantic Harp Seals, \textit{Pagophilus groenlandicus}. Canadian Science Advisory Secretariat Research Document 	2014/058.

A.D. Buren. 2012. Should we hunt seals to rebuild depleted fish stocks? The Osprey, 43(2): 40-43.

G.B. Stenson, M. Koen-Alonso \& A.D. Buren. 2009. Recent Advances on the Role of Seals in the Northwest Atlantic Ecosystem. NAFO Scientific Council Research Document 09/40.

A.D. Buren, M. Koen-Alonso, W.A. Montevecchi, J.T. Anderson, B. deYoung \& G.K. Davoren. 2006. Modeling trophic interactions between parental common murres and capelin off the northeast Newfoundland coast. ICES CM 2006 / L:05.

A.D. Buren. 2006. Seabird conservation: shifting the focus toward ecosystems. The Osprey, 37(3): 75-77.

\end{hangparas}

\section{INVITED SPEAKER}
\vspace{0.1in} 
\begin{hangparas}{.4in}{1}
	A.D. Buren, M. Koen-Alonso, F. Mowbray, B. Nakashima, G.B. Stenson, P. Pepin, N. Ollerhead, W.A. Montevecchi. 2012. Environmental regulation of capelin in the Northwest Atlantic. Northwest Atlantic Fisheries Organization Scientific Council and its Standing Committees Meeting. June 4. Dartmouth, NS, Canada.
	
	A.D. Buren, M. Koen-Alonso, F. Mowbray, B. Nakashima, G.B. Stenson, P. Pepin, N. Ollerhead, W.A. Montevecchi. 2012. Environmental regulation of capelin in the Northwest Atlantic. Identifying indicators for monitoring Arctic marine biodiversity in Canada. National Advisory Process (NAP). February 6-8, 2012. Winnipeg, MB, Canada
\end{hangparas}	

\pagebreak

\section{CONFERENCE \& WORKSHOP PRESENTATIONS}
\subsection{Oral Presentations}
\begin{hangparas}{.4in}{1}
G.B. Stenson, A.D. Buren, C. Abraham. 2017. Declining growth rates and condition impact reproductive rates in Northwest Atlantic harp seals. 22$^{nd}$ Biennial Conference on the Biology of Marine Mammals. October 22-27 2017. Halifax, NS, Canada.

C. Gomez,  J. Lawson, A-L Kouwenberg, H. Moors-Murphy, A.D. Buren, C. Fuentes-Yaco, E.  Marotte, A.S.M. Vanderlaan. 2016. Predicted distribution of right whales in eastern Canadian waters: identifying priority areas to enhance monitoring. North Atlantic Right Whale Consortium Annual Meeting. November 2-3, 2016. New Bedford, USA.

C. Gomez, J. Lawson, A. Wright, A.D. Buren, V. Lesage, D. Tollit. 2016. A systematic review and meta-analysis on the behavioural responses of wild marine mammals to man-made sounds: synthesis and recommendations for the future. 4$^{th}$ International Marine Conservation Congress. July 30 – August 30 2016. St John's, NL. Canada.

A.D. Buren, M. Koen-Alonso, G.B. Stenson. 2015. Bioenergetics–allometric model: relative contributions of seals, fisheries and food availability on the lack of recovery and dynamics of the northern cod stock. 2015 Northern Cod Framework Meeting. Regional Peer Review. November 30 – December 4 2015. St John's, NL.

C. Gomez, J. Lawson, A. Wright, A.D. Buren, V. Lesage, D. Tollit. 2015. Behavioural responses of marine mammals to anthropogenic sounds: an exploratory meta-analysis. 21$^{st}$ Biennial Conference on the Biology of Marine Mammals. December 13-18 2015. San Francisco, USA.

G.B. Stenson, A.D. Buren, M.O. Hammill and M. Koen-Alonso. 2015. Control of reproduction in Northwest Atlantic harp seals and the impact of climate change. 21$^{st}$ Biennial Conference on the Biology of Marine Mammals. December 13-18 2015. San Francisco, USA.

G.B. Stenson, A.D. Buren, M.O. Hammill and M. Koen-Alonso. The impact of climate change on reproduction and population dynamics of Northwest Atlantic harp seals, \textit{Pagophilus groenlandicus}. Arctic Frontiers. Climate and Energy. 18-23 January 2015. Troms{\o}, Norway.

G.B. Stenson, A.D. Buren and M. Koen-Alonso. The impact of changing climate on reproduction of northwest Atlantic harp seals, \textit{Pagophilus groenlandicus}. ICES Annual Science Conference. 15-19 September 2014. A Coru\~{n}a, Spain.

W.A. Montevecchi, A.D. Buren, C.M. Burke, G.K. Davoren, S. Garthe, A. Hedd, L. MacFarlane Tranquilla, P.M. Regular, G.J. Robertson, S. Wilhelm. Ocean Climate Influences on Marine Birds in the Northwest Atlantic. ICES Annual Science Conference. 15-19 September 2014. A Coru\~{n}a, Spain.

A.D. Buren, M. Koen-Alonso \& G.B Stenson. The role of harp seals, fisheries and food availability in driving the dynamics of northern cod in waters off Newfoundland, Canada. American Fisheries Society Annual Meeting. 17-21 August 2014. Qu\'{e}bec City, QC, Canada.

A.D. Buren, M. Koen-Alonso, P. Pepin, F. Mowbray, B. Nakashima, G.B. Stenson, N. Ollerhead \& W.A. Montevecchi. Bottom-up regulation of capelin, a keystone forage species in the Northwest Atlantic. American Fisheries Society Annual Meeting. 17-21 August 2014. Qu\'{e}bec City, QC, Canada.

M.J. Morgan, R.M. Rideout. A.D. Buren, D Maddock Parsons \& M. Koen-Alonso. 2013. Are low condition and growth possible contributors to the lack of recovery of Northern cod? ICES/NAFO Symposium. Gadoid Fisheries: The Ecology and Management of Rebuilding. 15-18 October 2013. St Andrew's, NB, Canada

G.B. Stenson, M.O. Hammill, A.D. Buren. 2013. Direct and Indirect Impacts of Ice on Harp Seals in the Northwest Atlantic. Arctic Frontiers. Geopolitics and Marine Production in a Changing Arctic. 20-25 January 2013. Troms{\o}, Norway

A.D. Buren, M. Koen-Alonso, G.B. Stenson. 2012. Drivers of northern cod trajectory: using a simple dynamic model as a tool for testing alternative competing hypotheses. Ecosystem Research Initiative (ERI) – NEREUS Program. Regional Advisory Process (RAP). January 17-19, 2012. St John's, NL, Canada

A.D. Buren, M. Koen-Alonso, F. Mowbray, B. Nakashima, G.B. Stenson, P. Pepin, N. Ollerhead, W.A. Montevecchi. 2012. Environmental regulation of capelin. Ecosystem Research Initiative (ERI) – NEREUS Program. Regional Advisory Process (RAP). January 17-19, 2012. St John's, NL, Canada

M. Koen-Alonso, F. Mowbray, P. Pepin, N. Wells, D. Holloway, B. Vaters, A.D. Buren, J. Morgan, B. Brodie. 2012. Changes in the structure of the marine fish community in Newfoundland and Labrador waters in the period 1980-2010. Ecosystem Research Initiative (ERI) – NEREUS Program. Regional Advisory Process (RAP). January 17-19, 2012. St John's, NL, Canada

A.D. Buren, M. Koen-Alonso, G.B. Stenson, F. Mowbray. 2011. Is competition with harp seals impeding the recovery of cod stocks off Newfoundland, Canada? 19$^{th}$ Biennial Conference on the Biology of Marine Mammals. 26 November - 2 December 2011. Tampa, FL, USA.

A.D. Buren, M. Koen-Alonso, F. Mowbray, B. Nakashima, G.B. Stenson, P. Pepin, N. Ollerhead. 2011. Environmental regulation of capelin in the Northwest Atlantic. Regional Advisory Process (RAP) for Atlantic Cod in NAFO Subdivision 3Ps. October 25-28, 2011. St John's, NL, Canada

A.D. Buren, M. Koen-Alonso, F. Mowbray, B. Nakashima, G.B. Stenson, P. Pepin, N. Ollerhead. 2011. Environmental regulation of capelin in the Northwest Atlantic. ICES Annual Science Conference. 19-23 September 2011. Gdansk, Poland.

A.D. Buren, M. Koen-Alonso and GB. Stenson. 2010. Ecosystem considerations for limit reference points and projections: exploring bottom-up and top-down drivers of northern cod. Newfoundland and Labrador Regional Atlantic Cod Framework Meeting: Reference Points and Projection Methods for Newfoundland cod stocks. November 22-26, 2010. St. John’s, NL, Canada

W.A. Montevecchi, C. Burke, A.D. Buren, P. Regular, A. Hedd, G. Davoren. 2010. Seabird and capelin interactions. Regional Advisory Process (RAP) for Capelin (Subarea 2 + Div. 3KL). October 26-28, 2010. St. John’s, NL, Canada. 

W.A. Montevecchi, P Regular, A.D. Buren, C Burke, D Fifield, A Hedd, L McFarlane-Tranquilla, \& E Wilson. 2010. The Eastern Canadian gill-net removal experiment: tracking the population responses of seabirds to the ground-fishery closure. 1$^{st}$ World Seabird Conference. Seabirds: Linking the Global Ocean. 7-11 September 2010. Victoria, BC, Canada.

W.A. Montevecchi, A.D. Buren, C Burke, D Fifield, A Hedd, L McFarlane-Tranquilla, P Regular \& E Wilson. 2010. The eastern Canadian gill-net removal experiment. Seabird population responses to the fishery closure. 24$^{th}$ International Congress for Conservation Biology. 3-7 July 2010. Edmonton, Alberta, Canada.

M. Koen-Alonso, F. Mowbray, P. Pepin, J. Morgan, B. Brodie, B. Vaters, D. Holloway, A.D. Buren, K. Dwyer, G. Stenson and K. Gilkinson. 2010. Key aspects of the Newfoundland-Labrador Shelf Ecosystem (NAFO Divs. 2J3KLNO). Zonal Advisory Process on Northern and Striped Shrimp. March 24 – 26, 2010 St. John’s, NL, Canada

A.D. Buren, M. Koen-Alonso \& G.B. Stenson. 2010. Modelling bottom-up and top-down drivers of Northern cod dynamics: Update on ongoing work. Regional Advisory Process for 2J3KL Cod (Cod RAP 2010). March 15-19 2010, St. John’s, NL

A.D. Buren, M. Koen-Alonso \& G.B. Stenson. 2009. An exploration of the roles of seal predation, fisheries and food availability in the lack of recovery of Atlantic cod in the waters off Newfoundland, Canada. 18th Biennial Conference on the Biology of Marine Mammals. October 2009. Qu\'{e}bec City, QC, Canada.

G.B. Stenson, M. Koen-Alonso \& A.D. Buren. 2009. Estimating diets and prey consumption by NW Atlantic harp seals in the waters of Newfoundland, Canada. 18th Biennial Conference on the Biology of Marine Mammals. October 2009. Qu\'{e}bec City, QC, Canada.

G.B. Stenson, M. Koen-Alonso, A.D. Buren \& D. McKinnon. 2009. Estimating prey consumption by NW Atlantic harp seals in Newfoundland waters. NAMMCO SC Working Group on Marine Mammals and Fisheries Interactions. Iceland, 15-17 April 2009

A.D. Buren, M. Koen-Alonso \& G.B. Stenson. 2009. Interactions between Northern cod and harp seals: A preliminary exploration of the roles of seal predation, capelin availability and fisheries. Zonal Scientific Review of Atlantic Cod Stocks in Eastern Canada (Cod ZAP 2009). Feb 24-Mar 06 2009, St. John’s, NL

M. Koen-Alonso \& A.D. Buren. 2009. Maximum Sustainable Yield (MSY) in a multispecies context: Generalities and illustrative analyses for 2J3KL cod. Zonal Scientific Review of Atlantic Cod Stocks in Eastern Canada (Cod ZAP 2009). Feb 24-Mar 06 2009, St. John’s, NL

M. Koen-Alonso, A.D. Buren \& G.B. Stenson. 2008. A multinomial regression approach to reconstruct diet composition. Seals Impact Workshop 2: International workshop to review the impacts of seals on Atlantic cod stocks in eastern Canadian waters, hosted by Fisheries and Oceans Canada. November 2008. Halifax, Nova Scotia, Canada.

A.D. Buren, M. Koen-Alonso, K. Dwyer \& G.B. Stenson. 2008. Comparative analysis of the diet of cod, turbot and harp seal in Newfoundland waters (1986-1996). Seals Impact Workshop 2: International workshop to review the impacts of seals on Atlantic cod stocks in eastern Canadian waters, hosted by Fisheries and Oceans Canada. November 2008. Halifax, Nova Scotia, Canada.

A.D. Buren, M. Koen-Alonso \& G.B. Stenson. 2008. Exploring predation and competition impacts of harp seal on Northern cod. Modeling approach and preliminary runs. Seals Impact Workshop 2: International workshop to review the impacts of seals on Atlantic cod stocks in eastern Canadian waters, hosted by Fisheries and Oceans Canada. November 2008. Halifax, Nova Scotia, Canada.

G.B. Stenson, M. Koen-Alonso, A.D. Buren \& D. McKinnon. 2008. Estimating prey consumption by NW Atlantic harp seals in Newfoundland waters. Seals Impact Workshop 2: International workshop to review the impacts of seals on Atlantic cod stocks in eastern Canadian waters, hosted by Fisheries and Oceans Canada. November 2008. Halifax, Nova Scotia, Canada.

A.D. Buren, M. Koen-Alonso \& G.B. Stenson. 2008. Reconstructing diet composition using a multinomial regression approach. ICES/NAFO/NAMMCO Symposium on the Role of Marine Mammals in the Ecosystem in the 21$^{st}$ Century. Dartmouth, Nova Scotia, Canada. 29 September – 1 October.

A.D. Buren, M. Koen-Alonso, KS Dwyer \& G.B. Stenson. 2008. Is there room for competition among fish top predators and harp seals in the Northwest Atlantic (NAFO Div. 2J3KL)?. ICES/NAFO/NAMMCO Symposium on the Role of Marine Mammals in the Ecosystem in the 21$^{st}$ Century. Dartmouth, Nova Scotia, Canada. 29 September – 1 October.

G. B. Stenson, M. Koen-Alonso and A. D. Buren. 2008. Recent Advances on the Role of Seals in the Northwest Atlantic Ecosystem. ICES/NAFO/NAMMCO Symposium on the Role of Marine Mammals in the Ecosystem in the 21$^{st}$ Century. Dartmouth, Nova Scotia, Canada. 29 September – 1 October.

KS Dwyer, A.D. Buren \& M. Koen-Alonso. 2008. Diet of Greenland halibut in the Newfoundland Shelf, Northwest Atlantic, during 1978-2003: reflections of a changing ecosystem. 7$^{th}$ International Flatfish Symposium. Sesimbra, Portugal. 2-7 November.

Hedd., P.M.Regular, W.A.Montevecchi, C.Burke, A.D.Buren \& D.A.Fifield. 2008. Going deep: Common Murres dive in sub-0$^{o}$C water for aggregated slow-moving capelin. Pacific Seabird Group Conference, Blaine, Washington, 27 February - 2 March

C.M.Burke, W.A. Montevecchi, M. Koen-Alonso, P. Penton \& A.D. Buren. 2008. Flexible foraging by murres in the face of changing capelin availability. Pacific Seabird Group Conference, Blaine, Washington, 27 February - 2 March.

A.D. Buren, M. Koen-Alonso \& G.B. Stenson. 2007. Harp seal diet in NL, multinomial regression approach. Seals Impact Workshop 1: International workshop to review the impacts of seals on Atlantic cod stocks in eastern Canadian waters, hosted by Fisheries and Oceans Canada. November 2007. Halifax, Nova Scotia, Canada.

M. Koen-Alonso, A.D. Buren, A. Bundy \& G. Stenson. 2007. Bionenergetic-allometric multispecies models: approach description and ongoing work in the Newfoundland-Labrador and Eastern Scotian Shelf systems. Seals Impact Workshop 1: International workshop to review the impacts of seals on Atlantic cod stocks in eastern Canadian waters, hosted by Fisheries and Oceans Canada. November 2007. Halifax, Nova Scotia, Canada.

M. Koen-Alonso \& A.D. Buren. 2006. Top predators' diet as indicator of ecosystem status: what do we need to make it work? DFO National Science Workshop. Mont-Joli, Qu\'{e}bec, November 21-23, 2006.

A.D. Buren, M. Koen-Alonso, W.A. Montevecchi, J.T. Anderson, B. deYoung \& G.K. Davoren. 2006. Modeling trophic interactions between parental common murres and capelin off the northeast Newfoundland coast. ICES Annual Science Conference. 19-23 September 2006. Maastricht, the Netherlands.

A.D. Buren, M. Koen-Alonso, W.A. Montevecchi, J.T. Anderson, B. deYoung \& G.K. Davoren. 2006. Modeling the link between prey availability and diet: Common murre (\textit{Uria aalge}) and capelin (\textit{Mallotus villosus}) interaction during the breeding season around Funk Island. NAFO Symposium: Environmental and Ecosystem Histories in the Northwest Atlantic – What Influences Living Marine Resources? 13-15 September 2006. Dartmouth, Nova Scotia, Canada

A.D. Buren, M. Koen-Alonso, W.A. Montevecchi, J.T. Anderson, B. deYoung \& G.K. Davoren. 2006. Modeling trophic interactions between parental common murres and capelin off the northeast Newfoundland coast. In: Abstracts of the 2006 Fisheries and Marine Ecosystems (FAME) Graduate Student Conference: Integrating Science and Policy. Camp Alexandra, Crescent Beach, British Columbia, Canada.

A.D. Buren, M. Koen-Alonso \& G. Stenson. Predator-prey interaction between harp seals and Atlantic cod: An exploration of sources of variation. ICES/GLOBEC Workshop on the Decline and Recovery of Cod Stocks throughout the North Atlantic, including tropho-dynamic effects. May 9-12, 2006. St John's, Newfoundland, Canada.
 
A.D. Buren. Modeling trophic interactions between parental common murres and capelin off the northeast Newfoundland coast. Workshop of the NSERC Strategic Project 'Importance of capelin (Mallotus villosus) biology in sustaining trophic interactions in the Northwest Atlantic'. April 18-21, 2006. St John's, Newfoundland, Canada.

A.D. Buren. Modelling a top predator’s diet: the Common Murre. Workshop of the NSERC Strategic Project 'Importance of capelin (\textit{Mallotus villosus}) biology in sustaining trophic interactions in the Northwest Atlantic'. March 1-4, 2005. St John's, Newfoundland, Canada.

A.D. Buren, M. Koen-Alonso, S.N. Pedraza, E.A. Crespo, \& N.A. García. 2003. Cambios en la dieta de la raya picuda (\textit{Dipturus flavirostris}) en dos per\'{i}odos de los \'{u}ltimos diez a\~{n}os. In: Abstracts of the V Jornadas Nacionales de Ciencias del Mar. December 8-12, 2003, Mar del Plata, Argentina. [Translation of the title: Changes in the diet of the beaked skate (\textit{Dipturus flavirostris}) between two periods in the last ten years]
	
	
\end{hangparas}		
\subsection{Poster Presentations}
\begin{hangparas}{.4in}{1}
A.D. Buren, G.B. Stenson. 2017. Comparative diet analyses of four sympatric seal species, in waters off Newfoundland and Labrador Canada. 22nd Biennial Conference on the Biology of Marine Mammals. October 22-27 2017. Halifax, NS, Canada.

G.B. Stenson, M. Koen Alonso, A.D. Buren \& D. McKinnon. The diet of harp seals, \textit{Pagophilus groenlandica}, in a changing environment. 17th Biennial Conference on the Biology of Marine Mammals. November 2007. Cape Town, South Africa.

GK. Davoren, P. Penton, C. May, B. Reinfort, N. Record, B. deYoung, C. Burke, W.A. Montevecchi, D. Andrews, A.D. Buren, M. Koen-Alonso, J.T. Anderson, C. Rose-Taylor, T. Bell, and S. Garthe. 2006. The Importance of capelin (\textit{Mallotus villosus}) in the Northwest Atlantic. NAFO Symposium: Environmental and Ecosystem Histories in the Northwest Atlantic – What Influences Living Marine Resources? 13-15 September 2006. Dartmouth, Nova Scotia, Canada.

G.K. Davoren, C. May, P. Penton, N. Record, B. deYoung, C.B. Burke, W.A. Montevecchi, A.D. Buren, D. Andrews, C. Rose-Taylor, T. Bell, M. Koen-Alonso \& J.T. Anderson. 2005. Importance of capelin (\textit{Mallotus villosus}) biology in sustaining trophic interactions in the Northwest Atlantic. Canadian Conference for Fisheries Research (CCFFR), January 2005, Windsor, Canada.

M. F. Grandi, A.D. Buren, E.A. Crespo, N.A. García, G.M. Svendsen \& S.L. Dans. New record of a complete specimen of the Shepherd’s Beaked Whale, \textit{Tasmacetus shepherdi}, in Argentina and comments on the age determination. 'XIX Jornadas Argentinas de Mastozoolog\'{i}a' (19$^{th}$ Argentinean Conference of Mammalogy). Puerto Madryn, Argentina, November 2004.
	
	
\end{hangparas}	

\section{LECTURES, SEMINARS AND WORKSHOPS}
\vspace{0.1in} 

Co-Lead and Coordinator of the R Technical Working Group for an Ecosystem Approach, NL Region. Fisheries and Oceans Canada. Fall 2017.

Guest lecturer for the course "Fisheries Ecology".\\ Fisheries and Marine Institute of Memorial University of Newfoundland. February 2017. \\
This lecture was intended for an undergraduate audience.

Science advice in the context of a complex and changing ocean ecosystem. Northwest Atlantic Fisheries Center, Fisheries and Oceans. May 2016. \\
This seminar was intended for a scientific audience 

Guest lecturer for the course "Fisheries Ecology". \\Fisheries and Marine Institute of Memorial University of Newfoundland. February 2016. \\
This lecture was intended for an undergraduate audience.

Guest lecturer for the course "Fisheries Ecology". \\Fisheries and Marine Institute of Memorial University of Newfoundland. October 2014. \\
This lecture was intended for an undergraduate audience.

From Physics to Cod: Bottom-Up Regulation of the Newfoundland-Labrador Shelf Ecosystem. Northwest Atlantic Fisheries Center, Fisheries and Oceans. May 2012. \\
This seminar was intended for a scientific audience 

Introduction to R.\\ Introductory Workshop for Biology and Cognitive and Behavioural Ecology graduate students and faculty.\\ Memorial University, September 29, 2011. Facilitators: Alejandro Buren \& Paul Regular.

Guest lecturer for the course “Field and Lab Methods in Behavioural Ecology”.\\ Memorial University. March 26, 2010. \\
This lecture was intended for a graduate audience.

Introduction to R.\\ Introductory Workshop for Biology and Cognitive and Behavioural Ecology graduate students and faculty. \\Memorial University, March 23, 2009. Facilitators: Alejandro Buren \& Paul Regular.


Are harp seals impeding the recovery of the northern cod stock? An exploration of ideas.\\
Psychology Department Colloquium, Memorial University: November 25, 2009. \\
This seminar was intended for a scientific audience.

Guest lecturer for the course “Biology for Earth Sciences”. \\Memorial University. November 13, 2009 \\
This lecture was intended for an undergraduate audience.
	

\section{AWARDS}
\vspace{0.1in} 
\begin{hangparas}{.4in}{1}
2013. Dr Wilfred Templeman Memorial Scholarship\\
Memorial University

2012.Dr Wilfred Templeman Memorial Scholarship\\
Memorial University

2011. Society for Marine Mammalogy Travel Grant

2008. Albert George Hatcher Memorial Scholarship\\
Memorial University

2007. Fellow of the School of Graduate Studies\\
Memorial University

2007-2011. 
School of Graduate Studies Fellowship \\
Memorial University

2007-2011. School of Graduate Studies Merit Award\\
Memorial University

2006. University of British Columbia Travel Scholarship

2005-2006. School of Graduate Studies Fellowship\\
Memorial University

2005-2006. Scholarship for Graduate Studies\\
Organization of the American States (OAS)

2001-2003. Lodging Scholarship\\
University of Patagonia

2002-2003. Fellowship for Undergraduate Studies\\
Provincial Ministry of Education (Argentina)
\end{hangparas}

\pagebreak
\section{SOFTWARE PROFICIENCY}
\vspace{0.2in} 

R (programming language and software environment for statistical computing and graphics)

Fortran77 (programming language)

Microsoft Access (development and advanced use of relational databases)

WinBUGS, JAGS, Stan (Bayesian statistical inference)

ADMB, TMB (non-linear statistical modeling software)

\LaTeX,  (Document preparation system)

R Markdown (authoring framework for data science)

GitHub (Version Control Software)

SAS (data management and statistics software)

Statistica, MiniTAB (statistics software)

ArcMap, Surfer (GIS mapping tools)



\section{REVIEWER FOR FOLLOWING JOURNALS}
\vspace{0.2in}
Ecology

Canadian Journal of Fisheries and Aquatic Sciences

Global Change Biology

Proceedings of the Royal Society B

Journal of Applied Ecology

Marine Ecology Progress Series

ICES Journal of Marine Sciences

Journal of Avian Biology

Waterbirds

\section{MEMBERSHIPS IN PROFESSIONAL OR LEARNED SOCIETIES}
\vspace{0.2in}
\begin{hangparas}{.4in}{1}
2014-present. American Fisheries Society

2009-present. Society for Marine Mammalogy    

2009-present.Inter-American Network of OAS Scholars     \\
Organization of the American States

2010. Student Representative\\
Memorial University\\
Cognitive and Behavioural Ecology (CABE) Programme Chair Search Committee

2007. Statistical Society of Canada

2003-2004. Student Representative\\
University of Patagonia\\
Advisory Committee, Puerto Madryn Campus

2002. Student Representative\\
University of Patagonia\\
University residence\\~\\
Notes: The Zonal Advisory Committee is the governing body of the Puerto Madryn Campus. 
Student representatives are elected by their peers.

\end{hangparas}
\section{PROFESSIONAL COMMITTEES}
\vspace{0.2in}
\begin{hangparas}{.4in}{1}
2015 - Present. National Marine Mammal Peer Review Committee\\
Fisheries and Oceans Canada

2018. EG 03/04/05 Recruiting Board Member\\
Fisheries and Oceans Canada, Newfoundland Region

2016 - 2018. Animal Care Committee\\
Fisheries and Oceans Canada, Newfoundland Region


\end{hangparas}
\section{ACADEMIC AND PROFESSIONAL DEVELOPMENT}
\vspace{0.2in}
\begin{hangparas}{.4in}{1}
2017. Time Series Models for Ecologists\\ 
Presented by Dr. Andrew Parnell (University College Dublin)

2017. Multivariate analysis of spatial ecological data using R\\
Presented by Dr. Subhash. Lele (University of Alberta)

2017. Unmanned Aerial Vehicles Operators Course\\ 
ING Robotics

2016. Aircraft Ditching Course Fixed Wing \\
Falck Safety Service Canada

2016. Chemical Immobilization of Wildlife\\
The Canadian Association of Zoo \& Wildlife Veterinarians and the Canadian Wildlife Health Cooperative

2015. Introduction to Linear Mixed Effects Models, GLMM and MCMC with R\\
Presented by Dr. Alain Zuur and Dr Elena Ieno (Highland Statistics)

2015. Canadian Firearms Safety Course\\
College of the North Atlantic

2014. Online PBSmapping  Course\\
Fisheries and Oceans Canada 
Presented by Dr. Rowan Haigh and Dr Denis Chabot

2010. Introduction to AD Model Builder\\
Northwest Atlantic Fisheries Centre, Fisheries and Oceans Canada \\
Presented by Dr. Anders Nielsen from DTU-aqua (Denmark)

2009. Sun Application Tuning Seminar
Memorial University\\
Application Performance Optimization on Sun Systems. Sun Microsystems

2006. Introduction to Bayesian statistics and WinBugs\\
Memorial University and Fisheries and Oceans Canada 

2005. Generalized Linear Models\\
Memorial University\\
Presented by Dr. D.C. Schneider from Memorial University, Canada.

2004. The use of statistics in ecology: a critical view  \\
Centro Nacional Patag\'{o}nico (CENPAT), Argentina\\
Presented by Dr. Victor Cueto and Dr.Javier Lopez de Casenave from the University of Buenos Aires, Argentina.

\end{hangparas}

\section{VOLUNTEER POSITIONS}
\vspace{0.2in}
\begin{hangparas}{.9in}{1}
2011-2018. Scouts Canada	\\	
Beavers and Cubs Leader

2001-2004. Sailing instructor \\
Nautical Club\\
University of Patagonia
\end{hangparas}


\end{resume}

\end{document}
